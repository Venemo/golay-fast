\documentclass[11pt,a4paper,oneside]{report}             % Single-side

\usepackage{t1enc}
\usepackage[utf8]{inputenc}
\usepackage{amsmath}
\usepackage{amssymb}
\usepackage{enumerate}
\usepackage{graphics}
\usepackage{listings}
\usepackage{color}
\usepackage{fancyvrb}
\usepackage{anysize}
\usepackage{setspace}
\usepackage[center]{caption}

\usepackage{titlesec}
	\titleformat{\chapter}[hang] 
	{\normalfont\huge\bfseries}{\chaptertitlename\ \thechapter:}{1em}{} 

\usepackage[hidelinks]{hyperref}
\usepackage{textcomp}
\usepackage{graphicx}
\usepackage{svg}

\onehalfspacing

\def\doubleunderline#1{\underline{\underline{#1}}}
\def\dul#1{\doubleunderline{#1}}
\def\ul#1{\underline{#1}}
\newcommand{\vect}[2]{\begin{bmatrix} #1 \\ #2 \end{bmatrix}}
\setcounter{MaxMatrixCols}{20}

\begin{document}

% Title page ==================================================================================
\begin{titlepage}
\begin{center}
{\huge Fast decoding of systematic Golay-24 code } \\
\vspace{0.5cm}
{\small Description of the algorithm used on board the SMOG-P, ATL-1 and SMOG-1 satellites } \\
{\large Timur Kristóf } \\

\vfill
{\large \today}
\end{center}
\end{titlepage}

\chapter{Introduction}

The Golay forward error correction code was invented by Marcel Golay in 1949. Its most well-known
use was onboard the NASA Voyager satellites where they used the Golay(24, 12, 8) code. These days
we know codes with much better error correction capabilities, but the Golay code is still a good
choice in applications where the size of each packet is small and/or the available hardware
resources are too modest, so there is no chance to use more complicated coding schemes.

The topic of this study is the \emph{systematic Golay(24, 12, 8)} code, called \emph{Golay-24} for short.
The aim is to present a new algorithm which can quickly and efficiently
decode this coding scheme even on machines with limited compute capabilities,
such as embedded systems.

This study not only presents the algorithm but also proves its correctness.

The algorithm presented here is a \emph{hard-decision} algorithm, since most of the embedded hardware
that I've worked with make a hard decision about the received bits, so the aim was to make decoding
efficient on this class of hardware.
Further research needs to be done to determine if the ideas presented here can be utilized in soft-decision algorithms.

\section{Definitions}

\doubleunderline{A} is the Golay matrix.

\[
    \doubleunderline{A} = 
    \begin{bmatrix}
            1 & 0 & 0 & 1 & 1 & 1 & 1 & 1 & 0 & 0 & 0 & 1 \\
            0 & 1 & 0 & 0 & 1 & 1 & 1 & 1 & 1 & 0 & 1 & 0 \\
            0 & 0 & 1 & 0 & 0 & 1 & 1 & 1 & 1 & 1 & 0 & 1 \\
            1 & 0 & 0 & 1 & 0 & 0 & 1 & 1 & 1 & 1 & 1 & 0 \\
            1 & 1 & 0 & 0 & 1 & 0 & 0 & 1 & 1 & 1 & 0 & 1 \\
            1 & 1 & 1 & 0 & 0 & 1 & 0 & 0 & 1 & 1 & 1 & 0 \\
            1 & 1 & 1 & 1 & 0 & 0 & 1 & 0 & 0 & 1 & 0 & 1 \\
            1 & 1 & 1 & 1 & 1 & 0 & 0 & 1 & 0 & 0 & 1 & 0 \\
            0 & 1 & 1 & 1 & 1 & 1 & 0 & 0 & 1 & 0 & 0 & 1 \\
            0 & 0 & 1 & 1 & 1 & 1 & 1 & 0 & 0 & 1 & 1 & 0 \\
            0 & 1 & 0 & 1 & 0 & 1 & 0 & 1 & 0 & 1 & 1 & 1 \\
            1 & 0 & 1 & 0 & 1 & 0 & 1 & 0 & 1 & 0 & 1 & 1
    \end{bmatrix}
\]

\doubleunderline{G} is the generator matrix of the systematic Golay(24, 12, 8) code.

\[
    \doubleunderline{G} =
    \begin{bmatrix}
        \doubleunderline{I} \\
        \doubleunderline{A}
    \end{bmatrix}
\]

If \underline{m} is the message (which we want to encode), and the parity vector is \underline{p},
then the encoded code word is \underline{x}.
The Hamming weight of the code word is $w\{\ul{x}\} = w\{\ul{m}\} + w\{\ul{p}\}$.

\[
    \underline{x} = \begin{bmatrix}
        \underline{m} \\
        \underline{p}
    \end{bmatrix} = \doubleunderline{G} \cdot \underline{m} = \begin{bmatrix}
        \doubleunderline{I} \\
        \doubleunderline{A}
    \end{bmatrix} \cdot \underline{m} = \begin{bmatrix}
        \doubleunderline{I} \cdot \underline{m} \\
        \doubleunderline{A} \cdot \underline{m}
    \end{bmatrix} = \begin{bmatrix}
        \underline{m} \\
        \doubleunderline{A} \cdot \underline{m}
    \end{bmatrix}
\]

If the \underline{x} code word is transmitted through a noisy channel, the received word might
contain errors. The error vector is \underline{e} (the error vector in the received message 
is $\underline{e}_m$ and the error vector in the received parity is $\underline{e}_p$)
and the received word is \underline{r}.

\[
    \ul{r} = \ul{x} + \ul{e} = \vect{\ul{r}_m}{\ul{r}_p} = \vect{\ul{m} + \ul{e}_m}{\ul{p} + \ul{e}_p}
\]

The Golay-24 code is capable of correcting 3 or fewer
bit errors, and detecting 4 bit errors. So the cases we care about the most are when $w\{\ul{e}\} \leq 3$
or $w\{\ul{e}\} = 4$.

\chapter{Algorithm}

This chapter presents the decoding algorithm. We will make the assumption that $w\{\ul{e}\} \leq 4$,
so the received vector contains at most 4 bit errors. This is reasonable because the Golay(24, 12, 8)
code is only capable to correct at most 3 bit errors, and detect 4 bit errors. Since the minimal
Hamming distance between each code word is 8, in case of 4 bit errors it is possible to create the
received vector from multiple different code words, therefore it is not possible to tell which was
the original.

\section{Basics}

First a few basic considerations are presented on which the algorithm is based.

\subsubsection{Inverse and transpose of the Golay matrix}

The \dul{A} Golay matrix is its own inverse and transpose.

\[
    \dul{A}^T = \dul{A}
\]
\[
    \dul{A}^{-1} = \dul{A}
\]

Consequences:
\begin{itemize}
    \item \dul{A} is an invertable linear transformation, so if $\ul{p} = \dul{A} \cdot \ul{m}$, then
          $\dul{A} \cdot \ul{p} = \dul{A} \cdot \dul{A} \cdot \ul{m} = \dul{A} \cdot \dul{A}^{-1} \cdot \ul{m} = \ul{m}$.
          In other words, the message can be restored from the parity vector.
    \item We only need to store the \dul{A} matrix in memory, and we don't need to ever invert or transpose it.
\end{itemize}

\subsubsection{Parity of messages with 1 Hamming weight}

If $\ul{m} = \ul{I}_{i}$, then $\ul{p} = \dul{A} \cdot \ul{m} = \dul{A} \cdot \ul{I}_{i} = \ul{A}_{i}$.
So if the message is the same as the $i$th row of the identity matrix $\dul{I}$, then the parity vector is the
same as the $i$-th row of the Golay matrix $\dul{A}$.

Consequences:
\begin{itemize}
    \item When changing a single bit at $\ul{I}_{i}$ in the message, the equivalent change to the parity is $\ul{A}_{i}$.
    \item When changing a single bit at $\ul{I}_{i}$ in the parity, the equivalent change to the message is $\ul{A}_{i}$.
\end{itemize}

\subsubsection{Hamming weight of code words from non-zero Hamming weight messages}

If $\ul{m} \neq \ul{0}$, then $w\{ \ul{x} \} = w\{ \ul{m} \} + w\{ \ul{p} \} \geq 8$.

Proof:
We already know that the Hamming distance between every possible code word is at least 8.
(Not proven here, but it's a well-known fact about the Golay code.)
The $\ul{m} = \ul{0}$ message is encoded to the code word $\ul{x} = \ul{0}$, so
the minimal Hamming distance of every possible code word can only be at least 8 if
in case of $\ul{m} \neq \ul{0}$ the $w\{ \ul{x} \} \geq 8$, otherwise there
would be a code word $\ul{x}$ whose Hamming distance from the $\ul{0}$ code word is less than 8.

Consequence:
\begin{itemize}
    \item $\forall \ul{m} \neq \ul{0} : w\{ \dul{A} \cdot \ul{m} \} \geq 8$
\end{itemize}

\subsubsection{Lemma 1.}

If $w\{\ul{e}\} \leq 4$ and $w\{\dul{A} \cdot \ul {r_m} + \ul{r}_p\} \leq 3$,
then $w\{\ul{e}_m\} = 0$ (which is equivalent to $\ul{e}_m = \ul{0}$).
So, assuming 4 or fewer bit errors, if we calculate the parity from the received message, and
the Hamming distance between the calculated parity and the received parity is less than
or equal to 3, then the received message is error-free.

Proof:

Let's suppose that the statement is false, and $\ul{e}_m \neq \ul{0}$.
The given expression looks like:

\[
    \dul{A} \cdot \ul{r}_m + \ul{r}_p =
    \dul{A} \cdot ( \ul{m} + \ul{e}_m ) + ( \ul{p} + \ul{e}_p ) =
    \dul{A} \cdot \ul{m} + \dul{A} \cdot \ul{e}_m + \dul{A} \cdot \ul{m} + \ul{e}_p =
    \dul{A} \cdot \ul{e}_m + \ul{e}_p
\]

Consider the code word generated from $\ul{e}_m$. The minimum Hamming distance between every possible code word is 8, therefore
in case of $\forall \ul{e}_m \neq \ul{0}$ the following is true:
$w\{ \ul{e}_m \} +  w\{ \dul{A} \cdot \ul{e}_m \} \geq 8$.
(See "Hamming weight of code words from non-zero Hamming weight messages" above.)

Let's define $w\{ \ul{e}_m \} = N$, then because $w\{ \ul{e} \} \leq 4$ the following
will also be true: $w\{ \ul{e}_p \} \leq 4 - N$.
So, $w\{ \dul{A} \cdot \ul{e}_m \} \geq 8 - N$.

So according the above, we get this:
\[
    w\{ \dul{A} \cdot \ul{e}_m + \ul{e}_p \} \geq (8 - N) - (4 - N) = 4
\]

What we got is that when $\ul{e}_m \neq \ul{0}$, then
$w\{ \dul{A} \cdot \ul{e}_m + \ul{e}_p \} \geq 4$ which is a contradiction, because
we started with the condition that it's $\leq 3$.

So we can conclude that the statement is true, and $\ul{e}_m = \ul{0}$.

\subsubsection{Lemma 2.}

If $2 \leq w\{\ul{e}\} \leq 4$ and $w\{\dul{A} \cdot \ul{r}_m + \ul{A}_i + \ul{r}_p\} \leq 2$,
then $\ul{e}_m = \ul{I}_i$. So if we assume the number of errors to be at least 2 but at most 4,
and we correct a 1-bit error $\ul{e}_{m, i} = \ul{I}_i$ in the received message, then we calculate
the parity of the corrected message, and the Hamming distance between the calculated parity and
the received parity is less than or equal to 2, then the received message contains exactly 1 error,
and that error is the $i$th row of the identity matrix, which is what was corrected.

Proof.

Consider the $\ul{r}_2 = \vect{\ul{r}_{m,2}}{\ul{r}_{p,2}} = \vect{\ul{r}_m + \ul{I}_i}{\ul{r}_p}$ received word.

Because of Lemma 1, we know that
if $w\{ \dul{A} \cdot \ul{r}_{m,2} + \ul{r}_{p,2} \} \leq 3$, then $\ul{e}_{m,2} = \ul{0}$.

The above formula of $\ul{r}_2$ yields the following result:

\[
    \dul{A} \cdot \ul{r}_{m,2} + \ul{r}_{p,2} =
    \dul{A} \cdot ( \ul{r}_m + \ul{I}_i ) + \ul{r}_p =
    \dul{A} \cdot \ul{r}_m + \dul{A} \cdot \ul{I}_i + \ul{r}_p =
    \dul{A} \cdot \ul{r}_m + \ul{A}_i + \ul{r}_p
\]

The above conditions contain $w \{ \dul{A} \cdot \ul{r}_m + \ul{A}_i + \ul{r}_p \} \leq 2$,
so for the received word $\ul{r}_2$ the following is true: $\ul{e}_{m,2} = \ul{0}$.

Since $\ul{r}$ only differs from $\ul{r}_2$ by only 1 message bit, the error of the parity is
$\ul{e}_p = \ul{e}_{p,2}$. The differing message bit is at $\ul{I}_i$.
So if $\ul{e}_{m,2} = \ul{0}$, then $\ul{e}_m = \ul{I}_i$.

\subsubsection{Lemma 3.}

Same as Lemma 1, but the other way around:

If $w\{\ul{e}\} \leq 4$ and $w\{\dul{A} \cdot \ul {r_p} + \ul{r}_m\} \leq 3$,
then $w\{\ul{e}_p\} = 0$  (which is equivalent to $\ul{e}_p = \ul{0}$).
So, if we calculate the message from the received parity, and
the Hamming distance between the calculated message and the received message is less than
or equal to 3, then the received parity is error-free. (And the received message is
$\dul{A} \cdot \ul {r_p}$.)

This can be proved in the exact same way as Lemma 1.

\subsubsection{Lemma 4.}

Same as Lemma 2, but the other way around:

If $2 \leq w\{\ul{e}\} \leq 4$ and $w\{\dul{A} \cdot \ul{r}_p + \ul{A}_i + \ul{r}_m\} \leq 2$,
then $\ul{e}_p = \ul{I}_i$. So if we assume the number of errors to be at least 2 but at most 4,
and we correct a 1-bit error $\ul{e}_{p, i} = \ul{I}_i$ in the received parity, then we calculate
the message of the corrected parity, and the Hamming distance between the calculated message and
the received message is less than or equal to 2, then the received parity contains exactly 1 error,
and that error is the $i$th row of the identity matrix, which is what we corrected.

This can be proved in the exact same way as Lemma 2.

\section{Steps of the algorithm}

\subsubsection{Step 1.}

Consider the $\ul{r}$ received vector.
Split $\ul{r}$ into received message and received parity:

\[
    \ul{r} = \vect{\ul{r}_m}{\ul{r}_p}
\]

\subsubsection{Step 2.}

Calculate the parity of the received message part: $\ul{p}_2 = \dul{A} \cdot \ul{r}_m$.

\subsubsection{Step 3.}

If $\ul{p}_2 = \dul{A} \cdot \ul{r}_m = \ul{p}$ then the received word contains no error, so we're done.

\subsubsection{Step 4.}

When the received message is error-free but the parity contains $w\{\ul{e}_p\} \leq 3$ errors, then according to
\emph{Lemma 1} we can tell this by looking at the Hamming distance between the received parity $\ul{r}_p$ and
the calculated parity $\ul{p}_2$ which is $d = w\{\ul{r}_p + \ul{p}_2\}$.

If $d \leq 3$ then the received message is error-free, and only the received parity contained at most 3 errors.

\subsubsection{Step 5.}

Calculate the message from the received parity, which will be $\ul{m_2} = \dul{A} \cdot \ul{r}_p$.

\subsubsection{Step 6.}

When the received parity is error-free but the received message has $w\{\ul{e}_m\} \leq 3$ errors, then according to
\emph{Lemma 2} we can tell this by looking at the Hamming distance between the received message $\ul{r}_m$
and the calculated message $\ul{m_2}$ which is $d = w\{\ul{r}_m + \ul{m_2}\}$.

If $d \leq 3$ then the received parity is error-free, and only the received message contained at most 3 errors.
So the decoded message is $\ul{m_2}$.

\subsubsection{Step 7.}

So far, the algorithm has covered every possible 1-bit error and all the 2 and 3-bit errors which were
limited to only the message or the parity. Let's now deal with the case where the received message contains 1 error,
and the received parity contains 1 or 2 errors.

For $\forall i : 1 \leq i \leq 12, \ul{e}_m = \ul{I}_i$ cases, let's look at the possible 1-bit errors of the
received message, and correct them in the following manner:

\[
    \ul{r}_{corr, i} = \vect{\ul{r}_{m, corr, i}}{\ul{r}_{p, corr, i}} = \vect{\ul{r}_m + \ul{I}_i}{\ul{r}_p + \ul{A}_i}
\]

Let's also calculate the parity of the corrected message:

\[
    \ul{p}_{i} = \dul{A} \cdot \ul{r}_{m, corr, i}
\]

There are 12 possible corrections, the right one is the case when:

\[
    w\{\ul{p}_{i} + \ul{r}_{p, corr, i}\} \leq 2
\]

Thanks to \emph{Lemma 2}, whichever $\ul{r}_{corr, i}$ satisfies the above formula, the correspoding $\ul{r}_{m, corr, i}$ will be
the decoded message.

\subsubsection{Step 8.}

Let's deal with the cases when the received parity has 1 error, and the received message has 1 or 2 errors.

For $\forall i : 1 \leq i \leq 12, \ul{e}_p = \ul{I}_i$ cases, let's look at the possible 1-bit errors of the
received parity, and correct them in the following manner:

\[
    \ul{r}_{corr, i} = \vect{\ul{r}_{m, corr, i}}{\ul{r}_{p, corr, i}} = \vect{\ul{r}_m + \ul{A}_i}{\ul{r}_p + \ul{I}_i}
\]

Let's calculate the message from the corrected parity:

\[
    \ul{m}_{i} = \dul{A} \cdot \ul{r}_{p, corr, i}
\]

There are 12 possible corrections, the right one is the case when:

\[
    w\{\ul{m}_{i} + \ul{r}_{m, corr, i}\} \leq 2
\]

Thanks to \emph{Lemma 4}, whichever $\ul{r}_{corr, i}$ satisfies the above formula, the corresponding $\ul{m}_{i}$  will be
the decoded message.

\subsubsection{Step 9.}

If the algorithm reached this point, it means that we've already covered all possible 1, 2 and 3 bit errors.
So we've detected a 4-bit error which we can't correct.

\end{document}


